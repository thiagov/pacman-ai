\documentclass[a4paper,12pt]{article}

\renewcommand{\baselinestretch}{1.0}


\usepackage[utf8]{inputenc}
\usepackage[brazil]{babel}
\usepackage{verbatim}
\usepackage{amssymb}
\usepackage{listings}
\usepackage[lined,algonl,ruled]{algorithm2e}
\usepackage{graphicx}
\usepackage{float}
\usepackage{subfig}
\usepackage{enumerate}
\usepackage[hmargin=3cm,vmargin=3cm]{geometry}

\lstset{stringstyle=\ttfamily,
language=C,
frame=single,
showstringspaces=false,
breaklines=true,
basicstyle=\footnotesize,
firstnumber=1,
numbers=left,
stepnumber=5,
numberstyle=\tiny,
xrightmargin=0pt,
xleftmargin=0pt}

\begin{document}

%\begin{titlepage}
%\begin{figure}[htb!]
%\begin{minipage}[c]{0.12\linewidth}
%\includegraphics[scale=0.85]{dcc.png}
%\end{minipage}
%\begin{minipage}[c]{0.8\linewidth}
%\begin{center}
%      {\large Universidade Federal de Minas Gerais - UFMG} \\
%      {\large Instituto de Ciências Exatas - ICEX} \\
%      {\large Departamento de Ciência da Computação - DCC}\\
%\end{center}
%\end{minipage}
%\end{figure}


%\begin{center}
%\vspace{5cm}
%{\Huge Projeto e Análise de Algoritmos\\}
%\vspace{0,8cm}
%{\LARGE \it Trabalho Prático 1 - Doors and Penguins\\}
%\vspace{0,8cm}
%\vspace{6cm}
%       {\tt \large Thiago Silva Vilela}
%\end{center}
%\end{titlepage}

%\newpage

%\newpage
%\tableofcontents

\newpage
\begin{center}
 {\LARGE  Trabalho Prático 2: Competição Pac-Man\\}
 {\Large \it Inteligência Artificial\\}
 {\Large Thiago Silva Vilela\\}
\end{center}
\vspace{1cm}

\section{Introdução}

O trabalho prático consiste em desenvolver agentes inteligentes para uma variação
competitiva do jogo Pac-Man. Nesse jogo, controlamos dois agentes em um labirinto
com dois territórios. Os dois agentes devem colaborar para defender seu território
e atacar o território dos adversários, pegando o maior número possível de
\textit{pacdots} do inimigo.

\section{Classificação do problema}
De acordo com o livro texto, o problema pode ser classificado da seguinte forma:

\begin{itemize}
  \item \textbf{Parcialmente observável}, uma vez que o agente não possui conhecimento
  do estado completo do ambiente. Mais especificamente, o agente não conhece a posição
  de seus inimigos, a não ser que eles estejam suficientemente próximos.
  \item \textbf{Multi-agente}, uma vez que certo agente competirá com dois inimigos
  por uma maior pontuação, e contará com um aliado para ajudá-lo a maximizar sua
  pontuação. O ambiente é multi-agente competitivo e multi-agente cooperativo.
  \item \textbf{Determinístico}, uma vez que o próximo estado do ambiente pode ser
  previsto através do estado atual e da ação escolhida pelo agente. Vale lembrar que,
  na definição de ambientes determiníticos usada pelo livro texto, ignoramos as
  incertezas que aparecem somente pelas ações dos outros agentes. O ambiente poderia
  ser considerado estocástico caso levamos essas incertezas em consideração.
  \item \textbf{Sequencial}, uma vez que a escolha de certa ação em um dado momento
  afeta todas as decisões futuras.
  \item \textbf{Semi-dinâmico}, uma vez que o agente tem um tempo fixo ($1$ segundo)
  para tomar uma decisão e, durante esse tempo, o ambiente não se altera. Dessa forma,
  o agente sabe que o ambiente não muda enquanto ele está deliberando, mas precisa se
  preocupar com a passagem do tempo.
  \item \textbf{Discreto}, uma vez que o ambiente é discreto e existe um número
  finito de estados e ações distintas no problema.
\end{itemize}

\section{Modelagem dos agentes}
Nessa seção serão descritas as estratégias e algoritmos utilizados na implementação dos
dois agentes desenvolvidos. Um dos agentes foi feito para atacar o território inimigo
e comer \textit{pacdots}, enquanto o segundo foi feito para defender seus próprios
\textit{pacdots}.

\subsection{Agente Ofensivo}


\subsection{Agente Defensivo}

\section{Análise de complexidade}
\subsection{Agente Ofensivo}
\subsection{Agente Defensivo}

\section{Análise de desempenho e discussão dos resultados}

%\begin{center}
%\includegraphics[scale=0.4]{best_algo.png}
%\end{center}

\end{document}
